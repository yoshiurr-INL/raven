The development of high-fidelity codes has undergone a significant acceleration in the last years. The newly deployed codes are characterized by 
remarkable improvements in the approximation of the underline physics (high approximation order and reduced use of empirical correlations). This
 improved fidelity is generally accompanied by a larger computational effort (computation time increased). This issue represents an obstacle in the 
 application of traditional computational techniques of uncertainty quantification (UQ) and risk (RA) associated with the operation of any industrial 
 plant (e.g, a nuclear reactor).
In order to overcome these limitations, several techniques are under investigation and parallel development. In the RAVEN framework, these 
techniques heavily use the concept of Reduced Order Modeling in order to accelerate the Risk assessment and uncertainty quantification. The 
Reduced Order Models (ROMs) are mathematical representations of a system, used to predict a selected output space of a physical system. In other 
words, they are mathematical objects that can be ``trained'' in order to predict some selected Figure of Merits (FOMs) of a system, within its training 
domain.
The usage of the ROMs can represent a valid solution to speed up the UQ and RA process, overall if multiple ROMs are combined together in order 
to simulate all the physical aspects that can describe the system (e.g. neutronic FOMs, Thermal- hydraulic FOMs, etc). The aim of this paper is the 
description of a newly developed technique (RAVEN object) to assemble multiple ROMs, in particular, and any model (physical codes, post-
processing algorithms, etc.), in general. This new RAVEN object allows the user to assemble multiple models in an entity that has been named 
``EnsembleModel'', identifying the input/output connections, and, consequentially the order of execution and which sub-models can be executed in 
parallel.
The ``Ensemble-Model'' concept represents a key development in the RAVEN code, since it allows to perform an integrated PRA and UQ analysis 
without the need to ``interrogate'' the high-fidelity codes if not strictly necessary, determining a remarkable speed-up of the whole process.