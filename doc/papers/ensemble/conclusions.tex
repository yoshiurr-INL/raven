\section{Conclusions}
\label{sec:conclusions}
In this paper, a newly developed capability of the RAVEN code has been shown. Through the Ensemble-Model entity, RAVEN is able to combine multiple models (i.e. Simulation Codes, Reduced Order Models, etc.), constructing a pipe network in order to transfer information among them. The addition of the Picard’s iteration scheme lets the user solving combinations of models that resolve in non-linear systems. 
The paper shows the early results for implementation of an ensemble approach for the coupling of surrogate models representing a multi-physics problem. While this is an initial implementation, the developed structure seems to support current needs and will be eventually extended in the future in order to couple codes whose input/output space is represented by high-density fields (e.g. temperature profiles in each nodal kinetic zones, etc.). This capability builds a complex system representation, even when the original models were not coupled, but just coupled their surrogate. Two applications are relevant for reliability analysis: (1) the possibility to build surrogate representation of a complex system, starting from libraries of surrogate models for each component, and (2) software implementation present during the first stage of surrogate model coupling when responses are high-density fields. 
The Ensemble-Model capability in RAVEN is currently used to couple a fuel performance code (Bison) and the T-H system code RELAP5 in order to analyze LOCA scenarios.ion.
