%% 
%% Copyright 2007, 2008, 2009 Elsevier Ltd
%% 
%% This file is part of the 'Elsarticle Bundle'.
%% ---------------------------------------------
%% 
%% It may be distributed under the conditions of the LaTeX Project Public
%% License, either version 1.2 of this license or (at your option) any
%% later version.  The latest version of this license is in
%%    http://www.latex-project.org/lppl.txt
%% and version 1.2 or later is part of all distributions of LaTeX
%% version 1999/12/01 or later.
%% 
%% The list of all files belonging to the 'Elsarticle Bundle' is
%% given in the file `manifest.txt'.
%% 
%% Template article for Elsevier's document class `elsarticle'
%% with harvard style bibliographic references
%% SP 2008/03/01

\documentclass[preprint,12pt,authoryear]{elsarticle}

%% Use the option review to obtain double line spacing
%% \documentclass[authoryear,preprint,review,12pt]{elsarticle}

%% Use the options 1p,twocolumn; 3p; 3p,twocolumn; 5p; or 5p,twocolumn
%% for a journal layout:
%% \documentclass[final,1p,times,authoryear]{elsarticle}
%% \documentclass[final,1p,times,twocolumn,authoryear]{elsarticle}
%% \documentclass[final,3p,times,authoryear]{elsarticle}
%% \documentclass[final,3p,times,twocolumn,authoryear]{elsarticle}
%% \documentclass[final,5p,times,authoryear]{elsarticle}
%% \documentclass[final,5p,times,twocolumn,authoryear]{elsarticle}

%% For including figures, graphicx.sty has been loaded in
%% elsarticle.cls. If you prefer to use the old commands
%% please give \usepackage{epsfig}

%% The amssymb package provides various useful mathematical symbols
\usepackage{amssymb}
%% The amsthm package provides extended theorem environments
%% \usepackage{amsthm}

%% The lineno packages adds line numbers. Start line numbering with
%% \begin{linenumbers}, end it with \end{linenumbers}. Or switch it on
%% for the whole article with \linenumbers.
%% \usepackage{lineno}

\journal{Reliability Engineering \& System Safety}

\begin{document}

\begin{frontmatter}

%% Title, authors and addresses

%% use the tnoteref command within \title for footnotes;
%% use the tnotetext command for theassociated footnote;
%% use the fnref command within \author or \address for footnotes;
%% use the fntext command for theassociated footnote;
%% use the corref command within \author for corresponding author footnotes;
%% use the cortext command for theassociated footnote;
%% use the ead command for the email address,
%% and the form \ead[url] for the home page:
%% \title{Title\tnoteref{label1}}
%% \tnotetext[label1]{}
%% \author{Name\corref{cor1}\fnref{label2}}
%% \ead{email address}
%% \ead[url]{home page}
%% \fntext[label2]{}
%% \cortext[cor1]{}
%% \address{Address\fnref{label3}}
%% \fntext[label3]{}

\title{Multi-Unit PRA: a Simulation-Based Approach}

%% use optional labels to link authors explicitly to addresses:
%% \author[label1,label2]{}
%% \address[label1]{}
%% \address[label2]{}

\author{D. Mandelli, C. Parisi, A. Alfonsi, D. Maljovec, S. St Germain, R. Boring, S. Ewing, C. Smith}

\address{Idaho National Laboratory (INL), 2525 Fremont Ave, 83402 Idaho Falls (ID), USA}

\author{M. Rasmussen}

\address{Norwegian University of Science and Technology (NTNU)}

\begin{abstract}
Dynamic Probabilistic Risk Analysis (PRA) methods couple stochastic methods 
(e.g., RAVEN, ADAPT, ADS, MCDET) with safety analysis codes (e.g., RELAP5-3D, 
MELCOR, MAAP) to determine risk associated to complex systems such as nuclear 
plants. Compared to classical PRA methods, which are based on static logic 
structures (e.g., Event-Trees, Fault-Trees), they can evaluate with higher 
resolution the safety impact of timing and sequencing of events on the 
accident progression. Recently, special attention has been given to nuclear 
plants which consist of multiple units and, in particular, on the safety 
impact of system dependencies, shared systems and common resources on core 
damage frequencies. In the literature, while classical PRA methods have been 
employed to model multi-unit plants, Dynamic PRA methods have never been applied 
to analyze a full multi-unit model. This paper presents a PRA analysis of a 
multi-unit plant using Dynamic PRA methods. We employ RAVEN as stochastic 
method coupled with RELAP5-3D. The plant under consideration consists of the 
three units and their associated spent fuel pools (SFPs). The studied initiating
event is a seismic induced station blackout event. We will describe in detail 
how the multi-unit plant has been constructed and, in particular, how unit 
dependencies and shared resources are modeled.
\end{abstract}

\begin{keyword}
%% keywords here, in the form: keyword \sep keyword
multi-unit \sep PRA \sep dynamic PRA \sep Reduced Order Modeling
\end{keyword}

\end{frontmatter}

%% \linenumbers

%% main text
\section{}
\label{}

\section{nomenclature}
\label{sec:nomenclature}
\section{Introduction}
\label{sec:introduction}

test citation~\cite{Nureg1150} and test figure~\ref{fig:raven}

\begin{figure}
    \centering
    \includegraphics[scale=0.5]{raven.pdf}
    \caption{RAVEN}
    \label{fig:raven}
\end{figure}

\section{rismc}
\label{sec:rismc}
\section{RAVEN}
\label{sec:raven}
 
The Risk Analysis and Virtual ENviroment 
(RAVEN
\footnote{Official website: \url{https://raven.inl.gov},\\ 
GITHUB repository: \url{https://github.com/idaholab/raven}})
~\cite{RAVEN_PSAM_2014,alfonsiEsrel2014} 
is a flexible and multi-purpose uncertainty quantification, regression analysis, probabilistic
risk assessment, data analysis and model optimization framework. Depending on the tasks to be 
accomplished and on the probabilistic characterization of the problem, RAVEN perturbs 
(e.g., Monte-Carlo, latin hypercube, reliability surface search) the response of the system 
under consideration by altering its own parameters. The system is modeled by third party software 
(e.g., RELAP5-3D~\cite{relap5}, MELCOR~\cite{Melcor}) and accessible to RAVEN either directly (software coupling) 
or indirectly (via input/output files). The data generated by the sampling process is analyzed using 
classical statistical and more advanced data mining approaches. RAVEN also manages the parallel dispatching 
(i.e. both on desktop/workstation and large High Performance Computing machines) of the software 
representing the physical model. RAVEN heavily relies on artificial intelligence algorithms to construct 
surrogate models of complex physical systems in order to perform uncertainty quantification, reliability 
analysis (limit state surface) and parametric studies.

By statistical analysis we include:
\begin{itemize}
  \item Sampling of codes, either stochastic (e.g., Monte-Carlo~\cite{DynamicReliabilityMonteCarlo} 
        and Latin Hypercube Sampling~\cite{LHShelton}, deterministic (e.g., grid and
        Dynamic Event Tree~\cite{AMENDOLAdylam,cojazziDylam}) or adaptive~\cite{ANS_S_2014_raven_LS,mandelliSVMANS}
  \item Generation of Reduced Order Models (ROMs)~\cite{ROM_Khalik}, also known as Surrogate models
  \item Post-processing of the sampled data and generation of statistical parameters (e.g., mean, 
        variance, covariance matrix).
\end{itemize}

Figure~\ref{fig:ravenScheme} shows a general overview of the elements that comprise the RAVEN statistical framework:

\begin{itemize}
  \item Model: it represents the pipeline between input and output space. It comprises both codes 
        (e.g., RELAP5-3D~\cite{relap5}) and also ROMs 
  \item Sampler: it is the driver for any specific sampling strategy (e.g., Monte-Carlo, LHS, 
        DET~\cite{ANS2014_adaptDET,PSA2013_Raven})
  \item Database: the data storing entity
  \item Post-processing module: the module that performs statistical analyses and visualizes results.
\end{itemize}

\begin{figure}
    \centering
    \centerline{\includegraphics[scale=0.6]{raven.pdf}} 
    \caption{Overview of RAVEN statistical framework components}
    \label{fig:ravenScheme}
\end{figure}

\subsection{Ensemble-Model Capabilities}
[ANDREA]
\section{Multi-unit modeling}
\label{sec:multiUnitModeling}

From a mathematical point of view, a single simulator run can be represented 
as a single trajectory in the phase space. The evolution of such a trajectory 
in the phase space as function of time $t$ can be described as follows:

\begin{equation}
    \frac{\partial \boldsymbol \theta }{\partial t}  = \boldsymbol \Xi (\boldsymbol \theta , \boldsymbol p, \boldsymbol s , t) 
    \label{eq:trajectory}
\end{equation}

where:
\begin{itemize}
  \item $\boldsymbol \theta = \boldsymbol \theta(t)$ represents the temporal 
        evolution of a simulated accident scenario, i.e., $\boldsymbol \theta(t)$ can 
        represent temperature inside the core of a PWR, the pressure level inside a containment
        building, the radionuclide concentration at a specific point outside the plant, etc.
  \item $\boldsymbol \Xi$ is the actual simulator code that describes how $\boldsymbol \theta$ 
        evolves in time
  \item $\boldsymbol s = \boldsymbol s(t,\boldsymbol p)$ represents the status of components 
        and systems of the model (e.g., status of emergency core cooling system, AC system)
\end{itemize}

By using the RISMC approach, if Monte-Carlo sampling is chosen, the PRA analysis is performed 
by~\cite{BWR_SBO_Mandelli}:
\begin{itemize}
  \item Associating a probabilistic distribution function (pdf) to the set of uncertain 
        parameters $\boldsymbol p$ (e.g., timing of events)
  \item Performing stochastic sampling of the pdfs defined in Step 1
  \item Performing a simulation run given $\boldsymbol p$ sampled in Step 2, i.e., solve Eq.~\ref{eq:trajectory}
  \item Repeating Steps 2 and 3 $M$ times and evaluating user defined stochastic parameters such as 
        Core Damage (CD) probability $P_{CD}$ as
        \begin{equation}
            P_{CD} = \frac{M_{CD}}{M} 
            \label{eq:CDprobability}
        \end{equation}
        where $M_{CD}$ is the number of simulations that lead to CD. 
\end{itemize}

In a multi-unit type of scenario, the dynamic of each unit is not independent but it can actually 
interact with the other units. Example of interactions are: electrical cross-ties and shared plant 
resources such as portable AC generators

Equation~\ref{eq:trajectory} refers to a single unit plant site; when multiple units are considered 
then it is needed to track the temporal evolution of each unit: multiple $\theta$ needs to be evaluated 
(one for each unit). 
Assuming that a three-unit plant is considered, Eq.~\ref{eq:trajectory} now becomes as follows:

\begin{equation}
  \begin{matrix}
     \begin{dcases*}
       \frac{\partial \boldsymbol{\theta}_1}{\partial t}  = \boldsymbol{\Xi}_1 (\boldsymbol{\theta}_1 , \boldsymbol p, \boldsymbol{s}_1 , \boldsymbol{s}_2 , \boldsymbol{s}_3, t)  \\     
       \frac{\partial \boldsymbol{\theta}_2}{\partial t}  = \boldsymbol{\Xi}_2 (\boldsymbol{\theta}_2 , \boldsymbol p, \boldsymbol{s}_1 , \boldsymbol{s}_2 , \boldsymbol{s}_3, t)  \\   
       \frac{\partial \boldsymbol{\theta}_3}{\partial t}  = \boldsymbol{\Xi}_3 (\boldsymbol{\theta}_3 , \boldsymbol p, \boldsymbol{s}_1 , \boldsymbol{s}_2 , \boldsymbol{s}_3, t)  \\       
     \end{dcases*}
  \end{matrix}
  \label{eq:MU_looseCoupled}
\end{equation}

Note that now the vector $s_i (i=1,\ldots,3)$ of each unit is shared among other units. This feature 
captures shared resources and possible system cross-ties among units.
In addition, intra-unit interactions such as a sub-set of human actions in a unit may be driven 
by the actual status of other unit (e.g., thermo-hydraulic limit and operational boundaries). 
Again, these actions may have cascade effects on the other units. This is particularly relevant 
for severe accident scenarios. Thus, now Eq.~\ref{eq:MU_looseCoupled} becomes:

\begin{equation}
  \begin{matrix}
     \begin{dcases*}
       \frac{\partial \boldsymbol{\theta}_1}{\partial t}  = \boldsymbol{\Xi}_1 (\boldsymbol{\theta}_1 , \boldsymbol{\theta}_2, \boldsymbol{\theta}_3 , \boldsymbol p, \boldsymbol{s}_1 ,
       \boldsymbol{s}_2 , \boldsymbol{s}_3, t)  \\
       \frac{\partial \boldsymbol{\theta}_2}{\partial t}  = \boldsymbol{\Xi}_2 (\boldsymbol{\theta}_1 , \boldsymbol{\theta}_2, \boldsymbol{\theta}_3 , \boldsymbol p, \boldsymbol{s}_1 ,
       \boldsymbol{s}_2 , \boldsymbol{s}_3, t)  \\
       \frac{\partial \boldsymbol{\theta}_3}{\partial t}  = \boldsymbol{\Xi}_3 (\boldsymbol{\theta}_1 , \boldsymbol{\theta}_2, \boldsymbol{\theta}_3 , \boldsymbol p, \boldsymbol{s}_1 ,
       \boldsymbol{s}_2 , \boldsymbol{s}_3, t)  \\
     \end{dcases*}
  \end{matrix}
  \label{eq:MU_tightCoupled}
\end{equation}

From a modeling point of view, solving Eq.~\ref{eq:MU_looseCoupled} or Eq.~\ref{eq:MU_tightCoupled} poses 
different challenges. Equation~\ref{eq:MU_looseCoupled} can in fact be solved by:
\begin{itemize}
  \item Sampling the set of uncertain parameters $\boldsymbol p$
  \item Determining the temporal profile of $\boldsymbol{s}_1$, $\boldsymbol{s}_2$, $\boldsymbol{s}_3$
  \item Run the simulator for each unit independently given $\boldsymbol p$, $\boldsymbol{s}_1$, $\boldsymbol{s}_2$, $\boldsymbol{s}_3$
\end{itemize}

On the other side, solving Eq.~\ref{eq:MU_tightCoupled} requires a system simulator that allows 
running the simulation of each unit simultaneously and sharing the variables
$\boldsymbol{\theta}_1$, $\boldsymbol{\theta}_2$, $\boldsymbol{\theta}_3$ among them.
This paper will focus on multi-unit case that can be described by Eq.~\ref{eq:MU_looseCoupled}. 




\section{Test Case}
\label{sec:testCase}

\subsection{Plant Description}
For our analysis we have chosen a 3-unit plant as shown in Fig.~\ref{fig:layout}. The chosen layout 
is not representative of any existing plant but it is simply fictitious.
From a topographical perspective, a large body of water is located in proximity of the NPP and it is 
employed as ultimate heat-sink for the plant.

\begin{figure}
    \centering
    \includegraphics[scale=0.5]{layout.pdf}
    \caption{Overview of the multi-unit plant}
    \label{fig:layout}
\end{figure}

All three units are composed by PWR systems (see Fig.~\ref{fig:PWRscheme}); the design of the PWR systems 
are identical for all the three units and it 
can be considered generic, i.e., it is not specific to an existing plant. 

\begin{figure}
    \centering
    \includegraphics[scale=0.5]{PWRscheme.pdf}
    \caption{Generic scheme of a PWR system}
    \label{fig:PWRscheme}
\end{figure}

\begin{figure}
    \centering
    \includegraphics[scale=0.5]{CCW-SW.pdf}
    \caption{Generic scheme of a the CCW and SW systems}
    \label{fig:CCW-SWscheme}
\end{figure}

For each PWR, the systems considered in the analysis are the following:
\begin{itemize}
  \item High Pressure Injection System (HPIS)
  \item Low Pressure Injection System (LPIS)
  \item Residual Heat Removal (RHR) system
  \item Accumulators (ACCs)
  \item Auxiliary Feed-water (AFW) system 
  \item Charging pumps
\end{itemize}

Special attention has been given to the design of the electrical and hydraulic systems:
\begin{itemize}
  \item The plant electrical system is shown in Fig.~\ref{fig:electricalScheme}. Two 
        electrical switch-yards can provide electrical power to all units. All units 
        have a set of Emergency Diesel Generators (EDGs)  and, in addition, a swing 
        EDG (i.e., EDGS) can be employed to provide an alternate AC power to either
        Unit 1 or Unit 2. Note also that the 6.6 KV emergency buses of Unit 1 and 
        Unit 2 can be cross-tied.
  \item The Auxiliary Feed-Water (AFW) system of Unit 1 and Unit 3 can be cross-tied. 
        Thus cooling to the secondary side can be provided from one unit to the other one.
  \item The Condensate Storage Tanks (CSTs) of Units 2 and Unit 3 can be cross-tied. 
        Thus the water source  for the secondary side of either unit can be used as
        water source for the other one. 
  \item Plant recovery crew is a shared resource within the plant. In case of severe 
        accident scenarios, Emergency Portable Equipments (EPEs) can be employed in order
        to restore water flow or AC power into the PWRs or SFPs. Each unit has its own 
        set of EPEs but it is here assumed that a single EPE team (i.e., plant recovery 
        crew) is present within the plant boundaries.
\end{itemize}

\begin{figure}
    \centering
    \centerline{\includegraphics[scale=0.6]{electricalScheme.pdf}}
    \caption{Plant electrical scheme}
    \label{fig:electricalScheme}
\end{figure}

Figure~\ref{fig:schemeDep} summarizes the dependencies among the three units.

\begin{figure}
    \centering
    \centerline{\includegraphics[scale=0.4]{schemeDep.pdf}}
    \caption{Summary of the dependencies among the three units}
    \label{fig:schemeDep}
\end{figure}

\subsection{Initiating Event}

The considered initiating event is a seismic event which causes the following events:
\begin{itemize}
  \item Both switch-yards are disabled
  \item All EDGs are disabled except EDGS which is initially aligned to Unit 2
  \item CST of Unit 2 has lost 80\% of its capacity 
  \item CST of Unit 3 is completely lost
  \item The seismic event might also rupture the SFPs. Thus a leak might be present during the accident scenario
\end{itemize}

The proposed accident scenario resembles a Station Black Out (SBO) event at the plant level except for the 
fact that the EDGS is the only source of AC power available and it can be directed toward either Unit 1 or Unit 2.

Prior the seismic event, the three units initial conditions are summarized in Table~\ref{tab:unitsStatus}:

\begin{table}
  \begin{tabular}{ | l | p{10cm} | }
    \hline      
      Unit 1 &  The PWR of Unit 1 (i.e., PWR1) is at full power (100\% power level) and it own SFP \\ \hline
      Unit 2 &  The PWR of Unit 2 (i.e., PWR2) is in mid-loop operation (i.e., shut-down mode) and it own SFP. 
                The mid-loop status is characterized by a primary coolant system drained to the 
                hot leg centerline and the existence of openings which a further reduction of 
                the mass inventory poses a serious risk, due to boil off and possible entrainment 
                or spill over of liquid\\ \hline
      Unit 3 &  The PWR of Unit 3 (i.e., PWR3) is at full power (108\%) that restarted a few weeks 
                earlier and its own SFP with a higher heat load since it contains used fuel recently 
                moved from the reactor. \\
    \hline  
  \end{tabular}
  \caption{Initial status of the three unit prior the accident scenario}
  \label{tab:unitsStatus}
\end{table}

\subsection{Accident progression}
\label{sec:accidentProgression}

Given the initiating event and the status of the plant, the analyzed accident focuses on the recovery strategy 
in order to place all PWRs and SFPs in a safe condition. For the scope of this analysis we consider a unit in
safe state when EPEs are connected to the unit (i.e., both PWR and SFP). These EPEs 
are located between the plant site boundaries and once connected to a unit can provide AC power and water injection.

An EPE is available for each unit and we assume that the the seismic event have not damaged the three EPEs available 
within the NPP. It is assumed that the EPE team can assist only one unit at a time, i.e., if three units need their 
own EPE, then the EPE team must first prioritize the units that require an immediate assistance.
  
In our case, even tough Unit 2 it is the one with AC power available it is in the most vulnerable situation since it 
is in mid-loop condition: low water inventory in the primary system and RPV head removed. Hence, in case or core damage
(CD) radioactive material will be directly released in the containment.

Unit 3 is the unit in the most critical condition since it is in SBO condition and heat removal from the RPV is limited
due to the fact that only 20\% of the CST inventory is available.

Unit 1 is in SBO condition like Unit 3 but it has more time margin before reaching CD due to the fact that core injection 
can be employed for several hours.

Given this assessment, three different strategies have been hypothesised as possible courses of action; these strategies
are described in detail in Sections \ref{sec:strategy1}, \ref{sec:strategy2} and \ref{sec:strategy3}.
For all these three strategies a temporal scheme is provided (see Fig.~\ref{fig:strategy1Scheme}, 
Fig.~\ref{fig:strategy2Scheme} and Fig.~\ref{fig:strategy3Scheme}). These schemes indicate the sequencing of events 
for all units. The initiating event (i.e., SBO condition) occurs at the top of each figure and accident progression 
occurs moving downward.

Section~\ref{sec:EDGSinvolAlign} introduce a disruptive element in the simulation: the involuntary alignment of the EDGs
from Unit 2 to Unit 1.


\subsubsection{Strategy 1}
\label{sec:strategy1}
This strategy prioritizes Unit 2 since it is in mid-loop condition. Hence the EPE team is initially directed toward
this unit. 
Once this task has been accomplished, Unit 3 becomes now the priority. In order to put this unit in a safe 
condition two parallel directions are followed:

\begin{itemize}
  \item Move the EPE team toward Unit 3 
  \item Align the EDGS from Unit 2 to Unit 1 so that also Unit 1 can be place in a safe condition and provide cooling 
        through AFW cross-tie between PWR 1 and PWR 3 (once completed). Note that the AFW cross-tie does not provide 
        cooling to the SFP
\end{itemize}

Finally, once Unit 3 EPE has been connected the EPE team move Unit 1. At this point Unit 1 should be already in a 
safe condition since EGSS has been aligned to Unit 1, this step has been added since the simulation run (for 
both PWR and SFP) stops when the EPE connected to the unit.

A temporal scheme of the temporal evolution of this accident scenario is shown in Fig.~\ref{fig:strategy1Scheme}.   

\begin{figure}
    \centering
    \centerline{\includegraphics[scale=0.7]{strategy1.pdf}}
    \caption{Sequence of events of recovery strategy 1}
    \label{fig:strategy1Scheme}
\end{figure}

\subsubsection{Strategy 2}
\label{sec:strategy2}

This strategy, similarly to Strategy 1, prioritizes Unit 2 since it is in mid-loop condition; thus, the EPE 
team is initially directed toward this unit. 
Once this task has been accomplished, Unit 3 becomes again now the priority. In order to put this unit in a safe 
condition two parallel directions are followed:

\begin{itemize}
  \item Move the EPE team toward Unit 3 
  \item Align the EDGS from Unit 2 to Unit 1 so that also Unit 1 can be place in a safe condition and provide 
        CST inventory from PWR 2 and PWR 3 (once completed). Note that the CST cross-tie does not provide cooling 
        to the SFP
\end{itemize}

Finally, once Unit 3 EPE has been connected the EPE team move Unit 1 as indicated also for Strategy 1.

A temporal scheme of the temporal evolution of this accident scenario is shown in Fig.~\ref{fig:strategy2Scheme}.  

\begin{figure}
    \centering
    \centerline{\includegraphics[scale=0.7]{strategy2.pdf}}
    \caption{Sequence of events of recovery strategy 2}
    \label{fig:strategy2Scheme}
\end{figure}
(see Fig.~\ref{fig:strategy2Scheme})


\subsubsection{Strategy 3}
\label{sec:strategy3}

This strategy prioritizes Unit 3; hence the EPE team is initially directed toward this unit. 
Once this task has been accomplished, Unit 1 becomes now the priority. In order to put this unit in a safe 
condition two parallel directions are followed: 

\begin{itemize}
  \item Move the EPE team toward Unit 1 
  \item Perform an AC cross-tie so that AC power generated by EDGS can be employed to provide power to both Unit
        1 and Unit 2. Note that it is here assumed that a correct AC management is implemented in order to avoid
        over-load of the EDGS.
\end{itemize}

Finally, once Unit 1 EPE has been connected the EPE team move Unit 2. At this point Unit 2 should be already in a 
safe condition since EGSS is still aligned to Unit 2, again, this step has been added since the simulation run (for 
both PWR and SFP) stops when the EPE connected to the unit.

A temporal scheme of the temporal evolution of this accident scenario is shown in Fig.~\ref{fig:strategy3Scheme}

\begin{figure}
    \centering
    \centerline{\includegraphics[scale=0.7]{strategy3.pdf}}
    \caption{Sequence of events of recovery strategy 3}
    \label{fig:strategy3Scheme}
\end{figure}

\subsubsection{EDGS involuntary alignment}
\label{sec:EDGSinvolAlign}

As part of the simulation we have introduced a stochastic event: the involuntary alignment of the EDGs.
This event can occur anytime during the simulation and it represents the involuntary alignment of the EDGS 
(initially aligned to Unit 2) to Unit 1. 

\hl{CHECK THIS WITH RON}

While this event can be considered an error or commission from an HRA perspective, from an operational point of view 
it does leave Unit 2 in an unsafe situation but it provide a safe condition to Unit 1. 

Note that now this stochastic element add an additional degree of freedom in the accident progression since the 
occurrence of this event changes the prioritization of the next unit to have the EPE connected.

As an example, Fig.~\ref{fig:strategy3SchemeInvolAlign} shows a possible sequence of events if Strategy 3 
(refer to Section~\ref{sec:strategy3}) has been chosen but with the involuntary alignment of EDGS while the EPE team
is working on Unit 3. Given the unexpected event, the EPE team moves to Unit 2 instead of Unit 1 since Unit 2
has higher priority. The difference sequence of events caused by involuntary alignment of EDGS can be viewed
by comparing Fig.~\ref{fig:strategy3Scheme} with Fig.~\ref{fig:strategy3SchemeInvolAlign}.

\begin{figure}
  \centering
  \centerline{\includegraphics[scale=0.7]{strategy3InvolAlign.pdf}}
  \caption{Sequence of events of recovery strategy 3 with involuntary alignment of EDGS}
  \label{fig:strategy3SchemeInvolAlign}
\end{figure}

\subsection{EPE actions}
\label{sec:EPEactions}

Depending on the unit status different actions can be employed when connecting EPEs to their respective unit.
\hl{CHECK THIS WITH CARLO}

\begin{table}
  \begin{tabular}{|l|l|l|p{5cm}|}
     \hline
                 & PWR1       & PWR2         & PWR3                                                                   \\ \hline \hline
     Strategy 1  & Refill CST & PS injection & Provide AC power and employ 1 LPIS and 1 HPIS or perform PS injection  \\ \hline
     Strategy 2  & Refill CST & PS injection & Perform PS injection or refill CST                                     \\ \hline
     Strategy 3  & Refill CST & PS injection & Provide AC power and employ 1 LPIS and 1 HPIS or perform PS injection  \\ \hline
  \end{tabular}
  \caption{EPE actions for each unit and for each recovery strategy}
  \label{fig:EPEactions}
\end{table} 


\section{RISMC approach to multi-unit modeling}
\label{sec:RISMC_MU_modeling}

\subsection{RAVEN Ensemble-Model Capabilities}
[ANDREA]

\subsection{System Models}
[CARLO]
\subsubsection{PWR1 and PWR3}

\subsubsection{PWR2}

\subsubsection{SFPs}

\subsection{Human models}
[RON]

\subsection{Plant Model}

\section{Multi-unit PRA}
\label{sec:multiUnitPRA}

Given the model described in Section~\ref{sec:RISMC_MU_modeling} and
the stochastic parameters listed in Table~\ref{tab:stochasticParameters} we have 
we have simulated about [???] accident scenarios using RAVEN Monte-Carlo sampling
capabilities. For each simulation, we have collected the output (OK or CD) from 
each model (i.e., all PWRs and SFPs).

The use of ROMs instead of the actual codes allowed us to generate a very large 
database of data which helped us to visualize timing and sequencing of events at 
the unit level.




\section{Plant ROM Modeling}
\label{sec:plantRomModeling}
[DAN]
\section{Plant Analysis Results}
\label{sec:plantAnalysisResults}

Historically the concept of core damage (CD) probability has been typically associated to a single unit. 
At a plant level, a separate value of CD probabaility can be asscoiated to all PWRs and SFPs. 
However, note that there is a high correlation among the six models of the plant (PWRs and SFPs). 
Thus it is also expectd that a high correlation among CD probabilities among the six models.
Thus, instaed of defining a single CD probability value for each PWR and SFP we define a probabiliuty 
value to a Plant Damage State (PDS) variable. This variable is a $6$-dimensional vector where each vector 
element describes the status of a plant model. For the scope of this paper we allowed two possible values 
for each element of the vector: OK or CD. Hence $2^6=64$ possible combinations are allowed.

The objective of this analysis is to rank PDSs based on their probability values.
Using a Monte-Carlo [14] sampling strategy we have simulated about 2000 accident scenarios. 
For each simualtion run we have performed the following steps:
\begin{enumerate}
  \item Sample a value for each stochastic parameter that is part of the set of parameters $p$ (e.g., timing of events)
  \item Performing the actual simulation run given $p$ sampled in Step 1 for each model of the multi-unit plant
  \item Collect the output (OK or CD) from each model and construct the PDS associated to the run
  \item Associate a unique probability value to the PDS accordingly to the chosen sampling
  \item Repeat Steps 1 trough 4, $M$ times 
\end{enumerate}
Group simulation runs based on their own PDS and evaluate probability associated to each of the 64 allowed PDSs.

\section{Results}
\label{sec:results}
The 1M samples have been post-processed by partitioning the data set in 64 subsets, a subset for each
PDS. 
For each subset/PDS a value of probability and error estimate associated to it have been determined 
(see Section~\ref{sec:plantAnalysisResults}).

Table~\ref{tab:resultsMain} summarize these findings and range the PDS based on their probability.
First of all, note that 14 out of 64 PDSs were actually generated; i.e., none of the 1M samples 
belong to 50 PDSs.

Secondarily, note that none of the recovery strategy are able to recovery PWR3: its condition at 
the beginning of the accident is the worst among the three units (lost of CST inventory on top of 
SBO condition). From separate calculations, PWR3 could be saved only if EPE3 would be connected
withing the first 50 minutes after SBO condition. Such condition, cannot be met given the boundary
conditions of the accident progression.

PWR1, on the other side, never reach CD condition: this is due to the fact that CST inventory is 
intact (compare to PWR3) and thus time required to reach CD condition is much longer. In addition,
PWR1 can be put in safe condition through several ways (see Section~\ref{sec:testCase}).

PWR2 and the SFPs appear to reach both CD and OK condition. The objective of the analysis is now 
to understand what are the driving factors behind each PDS instead of focusing only on PDS 
probabilities. Due to the complexity of PDS8, we will leave it at the end of the analysis.

\begin{table}
  \centering
  \begin{tabular}{c|cccccc|ccc}
    \hline
    ID & \multicolumn{6}{c}{PDS} & \multicolumn{3}{c}{Probability}   \\
    \cline{2-10}
         & PWR1 & PWR2 & PWR3 & SFP1 & SFP2 & SFP3 & mean & $5^{th}$ & $95^{th}$     \\
    \hline \hline
     8    & OK   & OK   & \cellcolor[gray]{0.95}CD   & OK   & OK   & OK   & 0.890199555 & 0.889684864 & 0.890713359 \\
      12   & OK   & OK   & \cellcolor[gray]{0.95}CD   & \cellcolor[gray]{0.95}CD   & OK   & OK   & 0.058915971 & 0.058529164 & 0.059303781 \\
      10   & OK   & OK   & \cellcolor[gray]{0.95}CD   & OK   & \cellcolor[gray]{0.95}CD   & OK   & 0.033966983 & 0.033669558 & 0.034265467 \\
      9    & OK   & OK   & \cellcolor[gray]{0.95}CD   & OK   & OK   & \cellcolor[gray]{0.95}CD   & 0.012604994 & 0.012422046 & 0.012789049 \\
      24   & OK   & \cellcolor[gray]{0.95}CD   & \cellcolor[gray]{0.95}CD   & OK   & OK   & OK   & 0.002102999 & 0.002028218 & 0.002178912 \\ 
      13   & OK   & OK   & \cellcolor[gray]{0.95}CD   & \cellcolor[gray]{0.95}CD   & OK   & \cellcolor[gray]{0.95}CD   & 0.001172999 & 0.001117271 & 0.001229862 \\
      14   & OK   & OK   & \cellcolor[gray]{0.95}CD   & \cellcolor[gray]{0.95}CD   & \cellcolor[gray]{0.95}CD   & OK   & 0.000581    & 0.00054194  & 0.000621195 \\    
      11   & OK   & OK   & \cellcolor[gray]{0.95}CD   & OK   & \cellcolor[gray]{0.95}CD   & \cellcolor[gray]{0.95}CD   & 0.000165    & 0.000144457 & 0.00018668  \\     
      26   & OK   & \cellcolor[gray]{0.95}CD   & \cellcolor[gray]{0.95}CD   & OK   & \cellcolor[gray]{0.95}CD   & OK   & 0.000156    & 0.000136041 & 0.000177095 \\
      28   & OK   & \cellcolor[gray]{0.95}CD   & \cellcolor[gray]{0.95}CD   & \cellcolor[gray]{0.95}CD   & OK   & OK   & 0.000111    & 9.43E-05    & 0.000128878 \\
      25   & OK   & \cellcolor[gray]{0.95}CD   & \cellcolor[gray]{0.95}CD   & OK   & OK   & \cellcolor[gray]{0.95}CD   & 1.10E-05    & 6.17E-06    & 1.70E-05    \\ 
     15   & OK   & OK   & \cellcolor[gray]{0.95}CD   & \cellcolor[gray]{0.95}CD   & \cellcolor[gray]{0.95}CD   & \cellcolor[gray]{0.95}CD   & 6.00E-06    & 2.61E-06    & 1.05E-05    \\     
     30   & OK   & \cellcolor[gray]{0.95}CD   & \cellcolor[gray]{0.95}CD   & \cellcolor[gray]{0.95}CD   & \cellcolor[gray]{0.95}CD   & OK   & 5.00E-06    & 1.97E-06    & 9.15E-06    \\         
     29   & OK   & \cellcolor[gray]{0.95}CD   & \cellcolor[gray]{0.95}CD   & \cellcolor[gray]{0.95}CD   & OK   & \cellcolor[gray]{0.95}CD   & 1.00E-06    & 5.13E-08    & 3.00E-06    \\    
    \hline
  \end{tabular}
  \caption{Multi-unit analysis results}
  \label{tab:resultsMain}
\end{table}

PDSs number 12, 10 and 9 are characterized by a single SFP in CD condition (on top of PWR3): SFP1, 
SFP2 and SFP3 respectively. The main driver is the loss of water inventory due to the seismic induced
SFP LOCA.
This conclusion might be obvious given the nature of the system; however, if we observe the 
histogram of the recovery strategy in each of these three PDSs 
(see Fig.~\ref{fig:histPDS_12_10_9_recoveryStrategy??}) we observe a pattern.
PDS12 and PDS9 are dominated mainly by samples that follows Strategy 1 and 2 while PDS10 is 
exclusively characterized by simulations that followed Strategy 3.
This is simply due to the fact that unit prioritization allows to recover only the SFP of the unit
that has EPE connected first. Heating-up of the SFP is so fast that does not allow for two consecutive 
EPE timings to occur.

\begin{figure}
  \begin{subfigure}{.5\linewidth}
    \centering
    \includegraphics[scale=0.3]{12_recoveryStrategy.pdf}
  \end{subfigure}%
  \begin{subfigure}{.5\linewidth}
    \centering
    \includegraphics[scale=0.3]{10_recoveryStrategy.pdf}
  \end{subfigure}\\[1ex]
  \begin{subfigure}{\linewidth}
    \centering
    \includegraphics[scale=0.3]{9_recoveryStrategy.pdf}
  \end{subfigure}
  \caption{Histograms of the variable recovery strategy for PDS12 (top left), PDS10 (top right) and PDS9 (bottom)}
  \label{fig:histPDS_12_10_9_recoveryStrategy}
\end{figure}

PDS24 is the first PDS that characterize an additional PWR to reach CD (on top of PWR3): PWR2. By looking at the
histogram of the input parameters (see Fig.~\ref{fig:histPDS_24}) that belong to this PDS we have identified 
that PWR2 reaches CD only if recovery strategy 3 is chosen. 
In addition, involuntary alignment of EDGS plays the major driver to reach PDS24. Interestingly, the time of such 
switch is also important: by looking at bottom histogram Fig.~\ref{fig:histPDS_24}, the distribution of the 
variable EDGSinvolAlignTime is characterized by two modes, an early mode and a late mode.
This feature is due to the fact that, in strategy 3, EDGS involuntary alignment 
(see Fig.~\ref{fig:strategy3SchemeInvolAlign}) might run in parallel with EPE3 or EPE1.
If this involuntary action occurs when EPE3 or EPE1 have just started, then PWR2 reach CD almost certainly due to
the PWR2 heat-up. If this involuntary action occurs when EPE3 or EPE1 are almost completed, then the EPE team
has time to prioritize Unit 2 and quickly recovery it.
The two modes of the bottom histogram of Fig.~\ref{fig:strategy3SchemeInvolAlign} correspond to an EDGS 
involuntary action that occurs right after EPE operation for Unit 3 (early mode) and for Unit 1 (late mode) has 
started.

\begin{figure}
  \begin{subfigure}{.5\linewidth}
    \centering
    \includegraphics[scale=0.3]{24_recoveryStrategy.pdf}
  \end{subfigure}%
  \begin{subfigure}{.5\linewidth}
    \centering
    \includegraphics[scale=0.3]{24_EDGSinvolAlign.pdf}
  \end{subfigure}\\[1ex]
  \begin{subfigure}{\linewidth}
    \centering
    \includegraphics[scale=0.3]{24_EDGSinvolAlignTime.pdf}
  \end{subfigure}
  \caption{PDS24: histograms of the variables recovery (top left), EDGSinvolAlign (top right) and EDGSinvolAlignTime (bottom)}
  \label{fig:histPDS_24}
\end{figure}

PDSs 13, 14 and 11 is a blend of PDS 12, 10 and 9: they contains 2 SFPs in CD condition. These PDS can be simply characterized
by the occurrence of 2 SFP LOCAs which are not correlated events; i.e., SFP LOCA have been modeled as independent events.
Thus, same conclusions derived from PDSs 9, 10 and 12, can be transposed for PDSs 13, 14 and 11.

PDSs 26, 28 and 25 are characterized by 1 SFP and PWR2 in CD condition; thus it represents a mix of PDS 24 and PDS 12, 10 and 9.
These PDS are in fact characterized by recovery strategy 3 and EDGS involuntary align along with a SFP LOCA.
Similarly to what has been presented for PDS 24, the interesting histogram is the one characterized EDGS involuntary time for 
these three PDSs (see~\ref{fig:histPDS_26_28_25_EDGSinvolAlignTime}). Note these histograms follow the same pattern 
of~\label{fig:histPDS_24} (bottom plot).

\begin{figure}
  \begin{subfigure}{.5\linewidth}
    \centering
    \includegraphics[scale=0.3]{28_EDGSinvolAlignTime.pdf}
  \end{subfigure}%
  \begin{subfigure}{.5\linewidth}
    \centering
    \includegraphics[scale=0.3]{26_EDGSinvolAlignTime.pdf}
  \end{subfigure}\\[1ex]
  \begin{subfigure}{\linewidth}
    \centering
    \includegraphics[scale=0.3]{25_EDGSinvolAlignTime.pdf}
  \end{subfigure}
  \caption{Histograms of the variable EDGSinvolAlignTime for PDS28 (top left), PDS26 (top right) and PDS25 (bottom)}
  \label{fig:histPDS_26_28_25_EDGSinvolAlignTime}
\end{figure}


\section{toolkit}
\label{sec:toolkit}

%% The Appendices part is started with the command \appendix;
%% appendix sections are then done as normal sections
%% \appendix

%% \section{}
%% \label{}

%% If you have bibdatabase file and want bibtex to generate the
%% bibitems, please use
%%
%%  \bibliographystyle{elsarticle-harv} 
%%  \bibliography{<your bibdatabase>}

%% else use the following coding to input the bibitems directly in the
%% TeX file.

\begin{thebibliography}{00}

%% \bibitem[Author(year)]{label}
%% Text of bibliographic item

\bibitem[ ()]{}

\end{thebibliography}
\end{document}

\endinput
%%
%% End of file `elsarticle-template-harv.tex'.
