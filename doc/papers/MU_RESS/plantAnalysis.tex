\section{Plant Analysis Results}
\label{sec:plantAnalysisResults}

Historically the concept of core damage (CD) probability has been typically associated to a single unit. 
At a plant level, a separate value of CD probabaility can be asscoiated to all PWRs and SFPs. 
However, note that there is a high correlation among the six models of the plant (PWRs and SFPs). 
Thus it is also expectd that a high correlation among CD probabilities among the six models.
Thus, instaed of defining a single CD probability value for each PWR and SFP we define a probabiliuty 
value to a Plant Damage State (PDS) variable. This variable is a $6$-dimensional vector where each vector 
element describes the status of a plant model. For the scope of this paper we allowed two possible values 
for each element of the vector: OK or CD. Hence $2^6=64$ possible combinations are allowed.

The objective of this analysis is to rank PDSs based on their probability values.
Using a Monte-Carlo [14] sampling strategy we have simulated about 2000 accident scenarios. 
For each simualtion run we have performed the following steps:
\begin{enumerate}
  \item Sample a value for each stochastic parameter that is part of the set of parameters $p$ (e.g., timing of events)
  \item Performing the actual simulation run given $p$ sampled in Step 1 for each model of the multi-unit plant
  \item Collect the output (OK or CD) from each model and construct the PDS associated to the run
  \item Associate a unique probability value to the PDS accordingly to the chosen sampling
  \item Repeat Steps 1 trough 4, $M$ times 
\end{enumerate}
Group simulation runs based on their own PDS and evaluate probability associated to each of the 64 allowed PDSs.
